\nonstopmode{}
\documentclass[a4paper]{book}
\usepackage[times,inconsolata,hyper]{Rd}
\usepackage{makeidx}
\usepackage[utf8]{inputenc} % @SET ENCODING@
% \usepackage{graphicx} % @USE GRAPHICX@
\makeindex{}
\begin{document}
\chapter*{}
\begin{center}
{\textbf{\huge Package `CQ2'}}
\par\bigskip{\large \today}
\end{center}
\inputencoding{utf8}
\ifthenelse{\boolean{Rd@use@hyper}}{\hypersetup{pdftitle = {CQ2: Objective Calibration of Baseflow and Quick-Slow CQ Models}}}{}
\begin{description}
\raggedright{}
\item[Type]\AsIs{Package}
\item[Title]\AsIs{Objective Calibration of Baseflow and Quick-Slow CQ Models}
\item[Version]\AsIs{0.1.0}
\item[Author]\AsIs{``Thomas Westfall''}
\item[Maintainer]\AsIs{The package maintainer }\email{yourself@somewhere.net}\AsIs{}
\item[Description]\AsIs{More about what it does (maybe more than one line)
Use four spaces when indenting paragraphs within the Description.}
\item[License]\AsIs{What license is it under?}
\item[Encoding]\AsIs{UTF-8}
\item[LazyData]\AsIs{true}
\item[RoxygenNote]\AsIs{7.3.1}
\end{description}
\Rdcontents{\R{} topics documented:}
\inputencoding{utf8}
\HeaderA{Chat1}{\code{Chat1} simple CQ model}{Chat1}
\keyword{model}{Chat1}
%
\begin{Description}
This is a model function. It predicts in-stream concentration.
\end{Description}
%
\begin{Usage}
\begin{verbatim}
Chat1(params, flow)
\end{verbatim}
\end{Usage}
%
\begin{Value}
A vector of concentration predictions for each time-step
\end{Value}
\inputencoding{utf8}
\HeaderA{hello}{Hello, World!}{hello}
%
\begin{Description}
Prints 'Hello, world!'.
\end{Description}
%
\begin{Usage}
\begin{verbatim}
hello()
\end{verbatim}
\end{Usage}
%
\begin{Examples}
\begin{ExampleCode}
hello()
\end{ExampleCode}
\end{Examples}
\printindex{}
\end{document}
